\documentclass[10pt,aspectratio=169]{beamer}
\usetheme[
%%% options passed to the outer theme
%    hidetitle,           % hide the (short) title in the sidebar
    hideauthor,          % hide the (short) author in the sidebar
%    hideinstitute,       % hide the (short) institute in the bottom of the sidebar
%    shownavsym,          % show the navigation symbols
%    width=2cm,           % width of the sidebar (default is 2 cm)
    hideothersubsections,% hide all subsections but the subsections in the current section
%    hideallsubsections,  % hide all subsections
    right               % right of left position of sidebar (default is right)
%%% options passed to the color theme
%    lightheaderbg,       % use a light header background
  ]{AAUsidebar}

% If you want to change the colors of the various elements in the theme, edit and uncomment the following lines
% Change the bar and sidebar colors:
%\setbeamercolor{AAUsidebar}{fg=red!20,bg=red}
%\setbeamercolor{sidebar}{bg=red!20}
% Change the color of the structural elements:
%\setbeamercolor{structure}{fg=red}
% Change the frame title text color:
%\setbeamercolor{frametitle}{fg=blue}
% Change the normal text color background:
%\setbeamercolor{normal text}{bg=gray!10}
% ... and you can of course change a lot more - see the beamer user manual.


\usepackage[utf8]{inputenc}
\usepackage[english]{babel}
\usepackage[T1]{fontenc}
\usepackage{booktabs}
\usepackage{multimedia}
\usepackage{multicol}
\usepackage{color}
\usepackage{listings}
\usepackage{listings-golang}
\usepackage{yaml}
\usepackage{tikz,pgfplots}
\usepackage{xcolor}
\definecolor{listinggray}{gray}{0.9}
\definecolor{lbcolor}{rgb}{0.9,0.9,0.9}
\lstset{
	backgroundcolor=\color{listbackground},
	tabsize=4,
	rulecolor=,
    frame=single,
	language=matlab,
        basicstyle=\scriptsize,
        upquote=true,
        aboveskip={1.5\baselineskip},
        columns=fixed,
        showstringspaces=false,
        extendedchars=true,
        breaklines=true,
        prebreak = \raisebox{0ex}[0ex][0ex]{\ensuremath{\hookleftarrow}},
        % frame=single,
        showtabs=false,
        showspaces=false,
        showstringspaces=false,
        identifierstyle=\ttfamily,
        keywordstyle=\color[rgb]{0,0,1},
        commentstyle=\color[rgb]{0.133,0.545,0.133},
        stringstyle=\color[rgb]{0.627,0.126,0.941},
}
\definecolor{feupblue}{RGB}{140,45,25}
\definecolor{listbackground}{HTML}{D8F0FB}
\definecolor{color6}{RGB}{242,130,65}
\definecolor{color1}{RGB}{146,212,245}
\definecolor{color2}{RGB}{78,184,237}
\definecolor{color3}{RGB}{2,154,229}
\definecolor{color4}{RGB}{1,108,160}
\definecolor{color5}{RGB}{0,54,92}
% Or whatever. Note that the encoding and the font should match. If T1
% does not look nice, try deleting the line with the fontenc.
\usepackage{helvet}

% colored hyperlinks
\newcommand{\chref}[2]{%
  \href{#1}{{\usebeamercolor[bg]{AAUsidebar}#2}}%
}

\title[Identificação de Tonalidade]% optional, use only with long paper titles
{Identificação de Tonalidade em ficheiros MIDI}
\subtitle{Trabalho de Grupo 3}  % could also be a conference name


\date{Maio 2021}

\author[Authors in sidebar] % optional, use only with lots of authors
{
  Diogo Remião 201706373 \\
  José Miguel Pinheiro 201705172
}
% - Give the names in the same order as they appear in the paper.
% - Use the \inst{?} command only if the authors have different
%   affiliation. See the beamer manual for an example

\institute[
%  {\includegraphics[scale=0.2]{aau_segl}}\\ %insert a company, department or university logo
  Sistemas Multimédia \\
  FEUP
] % optional - is placed in the bottom of the sidebar on every slide
{% is placed on the title page
  Sistemas Multimédia \\
  FEUP

  %there must be an empty line above this line - otherwise some unwanted space is added between the university and the country (I do not know why;( )
}


% specify a logo on the titlepage (you can specify additional logos an include them in
% institute command below
\pgfdeclareimage[height=1.5cm]{titlepagelogo}{AAUgraphics/feup_logo} % placed on the title page
%\pgfdeclareimage[height=1.5cm]{titlepagelogo2}{graphics/aau_logo_new} % placed on the title page
\titlegraphic{% is placed on the bottom of the title page
  \pgfuseimage{titlepagelogo}
%  \hspace{1cm}\pgfuseimage{titlepagelogo2}
}


\begin{document}
% the titlepage
{\aauwavesbg%
\begin{frame}[plain,noframenumbering] % the plain option removes the sidebar and header from the title page
  \titlepage
\end{frame}}
%%%%%%%%%%%%%%%%

% TOC
\begin{frame}{Agenda}{}
    \begin{multicols}{2}
        \setcounter{tocdepth}{2}
        \tableofcontents
    \end{multicols}
\end{frame}
%%%%%%%%%%%%%%%%
\chapter{Introdução} \label{sec:intro}

O objetivo de trabalho prende-se com a análise de \textbf{ferramentas de gestão e monitorização} de serviços de uma rede.
Ao contrário de ferramentas como MRTG e NTOP que monitorizam o tráfego, neste trabalho serão abordadas ferramentas que monitorizam diretamente os diferentes sistemas e os serviços neles alojados.

As ferramentas utilizadas são o \textbf{Nagios Core} e \textbf{Zabbix}, ambas grátis e open-source.
Será igualmente realizada uma análise às diferentes funcionalidades e capacidade de personalização de ambas as ferramentas num ambiente de teste criada no nossa bancada, onde serão alojados vários serviços nos diferentes computadores.
Testes de falha de sistemas e serviços serão efetuados de modo a analisar o funcionamento das ferramentas na deteção de falhas.

Por fim, será feita uma análise comparativa entre estas ferramentas com duas outras alternativas no mercado.


\section{Fundamentos Teóricos}
\begin{frame}{Fundamentos Teóricos}{}
\begin{itemize}
    \item Tonalidade
    \begin{itemize}
        \item Cifra
        \item Escalas / Acordes / Cadências
    \end{itemize}
    \item Tonalidades Próximas
    \begin{itemize}
        \item Relativa / Dominante / Sub-Dominante / Paralela
    \end{itemize}
    \item Técnicas de Identificação
\end{itemize}
\end{frame}

\subsection{Tonalidade}
\begin{frame}{Fundamentos Teóricos}{Tonalidade}
    \begin{columns}[]
        \begin{column}{.4\textwidth}
            \begin{itemize}
            \item O que é a tonalidade
            \item Cifra
            \item Derivações
            \begin{itemize}
                \item Escala
                \item Acordes
                \item Cadências
            \end{itemize}
            \end{itemize}
        \end{column}
        \begin{column}{.6\textwidth}
            \begin{figure}
                \includegraphics[width=.9\textwidth]{figs/escala.png}
            \end{figure}
            \vspace{-1cm}
            \begin{figure}
                \includegraphics[width=.9\textwidth]{figs/acorde.png}
            \end{figure}
        \end{column}
    \end{columns}
\end{frame}

\subsection{Tonalidades Próximas}
\begin{frame}{Fundamentos Teóricos}{Tonalidades Próximas}
    \begin{columns}[]
        \begin{column}{.5\textwidth}
            \begin{itemize}
            \item Tonalidades Próximas / Afastadas
            \item Tónica
            \item Relativa
            \begin{itemize}
                \item Maior
                \item Menor
            \end{itemize}
            \item Dominante
            \item Sub-Dominante e Relativas
            \end{itemize}
        \end{column}
        \begin{column}{.5\textwidth}
            \begin{figure}
                \includegraphics[width=.9\textwidth]{figs/modulacoes.png}
            \end{figure}
        \end{column}
    \end{columns}
\end{frame}

\subsection{Técnicas de Identificação}
\begin{frame}{Fundamentos Teóricos}{Técnicas de Identificação}
    \begin{columns}[]
        \begin{column}{.4\textwidth}
            \begin{itemize}
            \item Cifra
            \item Alterações (Sustenido e bemol)
            \item Progressões Harmónicas - Cadências
            \end{itemize}
        \end{column}
        \begin{column}{.6\textwidth}
            \begin{figure}
                \includegraphics[width=.95\textwidth]{figs/schubert.png}
            \end{figure}
        \end{column}
    \end{columns}
\end{frame}

\section{Método}
\begin{frame}{Método}{}
    \begin{itemize}
        \item Algoritmo de Krumhansl-Schmukler
        \item Key Profiles
        \begin{itemize}
            \item Krumhansl
            \item Variações: Temperley / Aarden / Bellman / Simple (Craig)
        \end{itemize}
        \item Exemplo
    \end{itemize}
\end{frame}

\subsection{Algoritmo de Krumhansl-Schmukler}
\begin{frame}{Método}{Algoritmo de Krumhansl-Schmukler}
    \begin{columns}[]
        \begin{column}{.5\textwidth}
            \begin{itemize}
                \item Baseado em perfis tonais (\textit{Key Profiles})
                \item Construção de uma distruibuição representativa da presença de cada nota
                \begin{itemize}
                    \item Temporal e variável com a métrica escolhida
                \end{itemize}
                \item Autocorrelação com cada perfil tonal
                \begin{itemize}
                    \item Perfil com maior correlação é o escolhido
                \end{itemize}
            \end{itemize}
        \end{column}
        \begin{column}{.5\textwidth}
            \begin{figure}
                \includegraphics[width=.9\textwidth]{figs/key_profiles_C.png}
            \end{figure}
        \end{column}
    \end{columns}
\end{frame}

\subsection{Key Profiles}
\begin{frame}{Método}{Key Profiles}
    \begin{itemize}
        \item Krumhansl-Schmukler \& Kessler
        \begin{itemize}
            \item Análise subjetiva com voluntários 
            \item Quão bem uma nota soa num elemento musical de uma tonalidade (Escala, Cadência, etc)
            \item Contrução de um perfil tonal para todas as tonalidades maiores e outra para as menores
        \end{itemize}
        \item Temperley
        \item Bellman e Aarden
        \item Simple (Craig)
    \end{itemize}
\end{frame}

\subsection{Exemplo}
\begin{frame}{Método}{Exemplo}
    \begin{columns}[]
        \begin{column}{.5\textwidth}
            \begin{itemize}
                \item Melodia "Yankee Doodle"
                \item Construção da distruibuição
                \item Correlação com os perfis de Krumhansl
                \item Melhor previsão - G Major (0.693)
                \begin{itemize}
                    \item D Major (0.485)
                    \item E Minor (0.398)
                    \item G Minor (0.394)
                \end{itemize}
            \end{itemize}
        \end{column}
        \begin{column}{.5\textwidth}
            \begin{figure}
                \includegraphics[width=.9\textwidth]{figs/yankee_song.png}
            \end{figure}
        \end{column}
    \end{columns}
\end{frame}
\section{Implementação}
\begin{frame}{Implementação}{}
\begin{itemize}
  \item Python Music21
  \item Data Set
\end{itemize}

\end{frame}

\subsection{Music21}
\begin{frame}[fragile]{Implementação}{Python Music21}
    \begin{columns}[]
      \begin{column}{.4\textwidth}
        \begin{itemize}
          \item Porquê Music21?
          \item Algoritmo de Krumhansl
          \item Vários perfis tonais
        \end{itemize}
    \end{column}
    \begin{column}{.6\textwidth}
        \begin{lstlisting}[language=Python, linewidth=0.90\linewidth,basicstyle=\tiny]
            score = music21.converter.parse('./bach/bwv_A_Major.midi')
            key = score.analyze('Krumhansl')
            print('Best Prediction = ', key.tonic.name, key.mode, key.correlationCoefficient)
            print('Second best Prediction = ', key.alternateInterpretations[0].tonic.name, key.alternateInterpretations[0].mode, key.alternateInterpretations[0].correlationCoefficient)
            print('Third best Prediction = ', key.alternateInterpretations[1].tonic.name, key.alternateInterpretations[1].mode, key.alternateInterpretations[1].correlationCoefficient)
            print('Worst Prediction = ', key.alternateInterpretations[22].tonic.name, key.alternateInterpretations[22].mode, key.alternateInterpretations[22].correlationCoefficient)
        \end{lstlisting}
    \end{column}
  \end{columns}
\end{frame}

\subsection{Data Set}
\begin{frame}{Implementação}{Data Set}
    \begin{itemize}
        \item \textit{Prelude \& Fugue} - J.S.Bach
        \begin{itemize}
            \item 24 peças
        \end{itemize}
        \item \textit{Études} - F.Chopin
        \begin{itemize}
            \item 18 peças
        \end{itemize}
        \item \textit{Piano Concertos 2 \& 3} - S.Rachmaninov
        \begin{itemize}
            \item 3 andamentos para cada concerto
        \end{itemize}
        \item Música Contemporânea
        \begin{itemize}
            \item Debussy
            \item Satie
            \item Ravel
        \end{itemize}
        \item Pop
    \end{itemize}
\end{frame}

\chapter{Análise de Resultados}

Nesta secção serão abordadas duas ferramentas de monitorização de tráfego, o \textbf{MRTG} e o \textbf{NTOP}.

\section{MRTG}

\subsection{Configuração}

O \textbf{MRTG} é uma ferramenta capaz de monitorizar o tráfego SNMP numa rede.
Através do MRTG, pode-se monitorizar o tráfego que entra e sai da nossa rede, com a informação apresentada em gráficos com diversos intervalos de tempo.
Deste modo, analisa-se de uma forma visual os padrões de tráfego da nossa rede.

Durante configuração do MRTG \cite{mrtg}, este foi configurado para que analisasse o tráfego no router de bancada (172.16.1.19).
Este router foi configurado de modo a que o serviço SNMP estivesse ativo e o MRTG pudesse obter essa informação.

\subsection{Propriedades da rede} \label{prop_rede}

\begin{figure}
    \centering
    \includegraphics[width=.6\linewidth]{figs/setup/network.png}
    \caption{Configuração da rede neste trabalho}
    \label{fig:network}
\end{figure}

Devido à configuração da rede no laboratório, não é obrigatório que todo o tráfego passe pelo router de bancada (Fig \ref{fig:network}).

Em primeiro lugar, o \textbf{tráfego local} faz uso do \textbf{switch}, logo esse tráfego não será registado pelo MRTG dado que não passa pelo router de bancada.
Isto inclui todo o tráfego entre todos os \textit{tux}s da sala I321.

Em segundo lugar, o router de bancada faz parte da mesma rede local que os \textit{tux}s, ligados pelo switch ao router de sala \textbf{firetux}.
Deste modo, não é obrigatório que um acesso de um tux a um host externo passe pelo router de bancada.
Foi preciso então configurar os \textit{tux}s, de modo a que utilizassem o router de bancada como \textit{default gateway} para forçar o tráfego a ir por esse caminho.
O router de bancada por sua vez tem o router de sala como o seu \textit{default gateway} (Fig \ref{fig:traceroute_1}).
Isto só funciona para novos queries, dado que da segunda vez que fazemos um acesso, o algoritmo irá determinar automaticamente que o melhor caminho é diretamente pelo \textbf{firetux} e mais uma vez o tráfego não irá ser registado (Fig \ref{fig:traceroute_2}).

\begin{figure}
    \centering
    \includegraphics[width=.8\linewidth]{figs/setup/trace_1.png}
    \caption{Primeiro traceroute de google.com}
    \label{fig:traceroute_1}
\end{figure}

\begin{figure}
    \centering
    \includegraphics[width=.8\linewidth]{figs/setup/trace_2.png}
    \caption{Segundo traceroute de google.com}
    \label{fig:traceroute_2}
\end{figure}

Para voltar a forçar o caminho, é preciso apagar o routing cache, que é feito pelo próprio sistema periodicamente.

Em terceiro lugar, é preciso notar que o tráfego que vem de host externo \textbf{nunca} passa pelo router de bancada, a menos que seja direcionado para o mesmo.
Isto faz sentido dado que router de bancada não faz parte do caminho ótimo para os \textit{tux}.

Levanta-se então a nível do tráfego que nos é possível monitorizar. 
Todo o tráfego que é gerado no sentido \textit{Local->Remote} é monitorizado.
Tráfego \textit{Remote->Local / Local->Local} não é possível de observar.
Não podemos registar, por exemplo, acessos feitos ao nosso servidor FTP, ou emails enviados para o nosso servidor email.
A sincronização entre o NTP server e NTP client também não é analisado dado que é tráfego local.

Uma solução seria monitorizar o tráfego no switch, onde todo o tráfego passa obrigatoriamente. Teria por consequência ver-se no entanto também tráfego local de outras bancadas.

\subsection{Análise temporal}

O MRTG apresenta gráficos temporais do tráfego que passa pelo router, quer a entrar, quer a sair.
Devido ao período de tempo em que este esteve a monitorizar, só faz sentido apresentar o gráfico diário e semanal (Fig \ref{fig:mrtg_main}).

\begin{figure}
    \centering
    \includegraphics[width=.6\linewidth]{figs/setup/mrtg.png}
    \caption{Gráficos temporais do tráfego no router MRTG}
    \label{fig:mrtg_main}
\end{figure}

Como se pode observar, o gráfico é constante e apresenta um padrão a nível de picos que está relacionado com a temporização dos cronjobs.
Estes cronjobs fazem diferentes tarefas como enviar mails, fazer "digs" no DNS e aceder a servidores HTTP.
Os picos maiores devem-se a testes que estavam ser executados durante a configuração o sistema, como, por exemplo, pelo envio de muitos emails num curto espaço de tempo.

Constatamos também que o tráfego que entra é praticamos igual ao tráfego que sai. 
Isto deve-se ao facto de muito pouco tráfego ser destinado ao próprio router, funcionando este apenas como uma \textit{gate}.

\section{NTOP}

\subsection{Configuração}

O \textbf{NTOP} é uma ferramenta de monitorização de tráfico. Oferece uma versão comunitária grátis, assim como versões profissionais de subscrição.

O NTOP funciona escutando o tráfego num adaptador de rede, i.e., interface. No nosso caso, é utilizada a interface \textit{eth0}. É de notar que o NTOP consegue monitorizar várias interfaces ao mesmo tempo.
Desse modo, este foi configurado para escutar essa interface \cite{ntop}. A web-app do NTOP é acedida através do link \verb|172.16.1.12:3000| e depois é efetuado o login com os dados configurados.

\subsection{Análise global}

Dentro da web-app do NTOP, após selecionar a interface desejada, é-nos apresentada uma página inicial onde se pode observar resumo do tráfego nessa interface: (Fig \ref{fig:ntop_main})
\begin{itemize}
    \item Monitorização do tráfego em tempo real no canto superior esquerdo
    \item Tipo de tráfego, com 86\% a ser \textit{Local->Remote}.
    \item Total de tráfego registado pelo NTOP, com 205 MB enviado e 41.6 MB recebidos. Dado que estão a ser realizados acessos FTP externos, ocorre um aumento considerável do total de MB enviados pelo servidor.
\end{itemize}

\begin{figure}
    \centering
    \includegraphics[width=.8\linewidth]{figs/setup/ntop_main.png}
    \caption{Main page da interface eth0}
    \label{fig:ntop_main}
\end{figure}

\subsection{Análise por tipologia}

Na tab "Apps", são apresentados 4 piecharts contendo a distribuição do tráfego por tipologia.
É apresentada uma avaliação total, assim como uma monitorização em tempo real (Fig \ref{fig:apps_charts}).

O \textit{FTP Data} corresponde a 52.4\% do tráfego, com o NTOP a representar 28.2\%.
Isto deve-se ao facto do NTOP gerar informação em realtime, que obriga a atualização constante da informação, gerando mais tráfego.
No momento em que foi tirado o print, não estava a ser detetado tráfego significativo de outros serviços, por isso o NTOP representa a maior fatia.

\begin{figure}
    \centering
    \includegraphics[width=.8\linewidth]{figs/setup/apps_charts.png}
    \caption{Piecharts da distribuição do tráfego por tipologia}
    \label{fig:apps_charts}
\end{figure}

Conseguimos também ver a distribuição total do tráfego em valores absolutos, onde podemos observar,
por exemplo, que o tráfego FTP representa a maior fatia de tráfego na interface (Fig \ref{fig:apps_table}).
Conseguimos também observar tráfego dos outros serviços instalados, nomeadamente DNS, HTTP, NTP e SMTP (Simple Mail Transfer Protocol).
O FTP estava a ser acedido para fazer download de um ficheiro de 250 Kb, pelo que gera mais tráfego.
O email por outro lado, era pequeno, pelo que gera menos tráfego.
Outros serviços são também detetados.


\begin{figure}
    \centering
    \includegraphics[width=.8\linewidth]{figs/setup/apps_tables.png}
    \caption{Distribuição do tráfego por tipologia}
    \label{fig:apps_table}
\end{figure}

\subsection{Análise temporal}

Tal como o MRTG, o NTOP também apresenta gráficos com a distribuição do tráfego a nível temporal.
Podemos observar diferentes intervalos temporais, como por exemplo, o horário (Fig \ref{fig:hourly}).
Podemos constatar neste gráfico vários picos periódicos de tráfego.
Estes correspondem, mais uma vez, aos pedidos FTP que geram muito mais tráfego que os restantes serviços.

\begin{figure}
    \centering
    \includegraphics[width=.8\linewidth]{figs/setup/hourly.png}
    \caption{Análise horária do tráfego}
    \label{fig:hourly}
\end{figure}

\subsection{Análise dos Hosts}

O NTOP apresenta também uma lista de hosts que utilizaram a interface para enviar/receber tráfego (Fig \ref{fig:hosts}).
Esta lista é dinâmica e apresenta apenas os hosts ativos recentemente, mas é possível fazer uma pesquisa por qualquer host que tenha utilizado a interface.
Podemos constatar que o \textit{tux12} (172.16.1.12) gerou a maior parte do tráfego (243.74 MB).
Isto faz sentido dado que foi neste \textit{tux} que se instalaram todos os serviços.

\begin{figure}
    \centering
    \includegraphics[width=.8\linewidth]{figs/setup/hosts.png}
    \caption{Lista dinâmica de hosts ativos}
    \label{fig:hosts}
\end{figure}

Também é possível fazer o mesmo tipos de análises que foram feitas para a interface, como a distribuição por tipologia e a distribuição temporal (Fig \ref{fig:hosts_apps}).

\begin{figure}
    \centering
    \includegraphics[width=.8\linewidth]{figs/setup/host_apps.png}
    \caption{Distribuição do tráfego por tipologia no tux12}
    \label{fig:hosts_apps}
\end{figure}

\subsection{Análise dos recursos do Sistema}

Além da análise do tráfego da interface, o NTOP apresenta também os recursos do sistema onde está instalado e o seu nível de utilização (Fig \ref{fig:system}).
Podemos observar gráficos de utilização do CPU do sistema para avaliar possíveis \textit{bottlenecks} a nível do hardware.
No nosso caso, os serviços instalados e a sua utilização não representam uma carga de utilização suficientemente grande para esta ferramenta ser útil neste projeto.

\begin{figure}
    \centering
    \includegraphics[width=.8\linewidth]{figs/setup/system.png}
    \caption{Utilização dos recursos do sistema}
    \label{fig:system}
\end{figure}

\section{Comparação}

Estas duas ferramentas e os resultados obtidos diferem em vários aspetos. Vamos analisar essas diferenças nesta secção.

\subsection{Monitorização de tráfego}

O NTOP, como mencionado anteriormente, regista todo o tráfego que usa a interface \textit{eth0}.
Esta interface é usada para praticamente todo tráfego, quer dentro da rede local quer externo.
Deste modo o NTOP é capaz de registar todo o tráfego dos serviços instalados no tux12.

O MRTG faz a monitorização do tráfego através do protocolo SNMP.
Ele foi configurado para monitorizar o router de bancada onde foi ativado o SNMP.
Pelas razões mencionadas na capítulo \ref{prop_rede}, o MRTG neste contexto não é capaz de ler todo o tráfego gerado pelos serviços, dado que nem todo o tráfego passa pelo router.
Desde modo, não é possível fazer uma análise total da utilização do nosso serviço com dois tipos de tráfego: \textit{Local->Local e Remote->Local}.

\subsection{User Interface}

O MRTG não faz distinção do tipo de tráfego.
De uma forma simples, apresenta gráficos com a evolução do tráfego total recebido e enviado ao longo do tempo.

O NTOP faz discriminação do tráfego, assim como da origem dele. Apresenta vários gráficos e tabelas onde podemos ver o tipo de tráfego, a sua origem, e a sua evolução ao longo do tempo.
Deste modo, podemos afirmar que o NTOP apresenta muita mais informação do que o MRTG.

Comparando os dois gráficos horários do NTOP (Fig \ref{fig:hourly}) e MRTG (Fig \ref{fig:mrtg_main}), pode-se observar que no MRTG não estão presentes os picos de tráfego em 100kB que vemos no NTOP.
Isto deve-se ao facto do tráfego FTP não estar a ser monitorizado. Estão a ser realizados acessos FTP do tipo \textit{Remote->Local}, um tipo de tráfego que não passa pelo router de bancada.
Para se poder observar tráfego FTP, seria necessário fazer ligações a servidores FTP da outra sala.
Isto não foi possível realizar em tempo útil, pois à data da recolha dos dados, não havia servidores FTP na outra sala disponíveis para este teste.

\subsection{Use Cases}
O NTOP provou ser a melhor ferramenta neste contexto. No entanto, não regista o tráfego utilizado por outras interfaces de outros \textit{tux}s.
Como na nossa configuração, todos os serviços estavam instados no \textit{tux12}, isto não foi um problema.

O MRTG é mais útil para monitorizar o tráfego de um servidor. Se estivéssemos a ler o switch, iríamos conseguir ver todo o tráfego \textit{Local->Local / Remote->Local / Local->Remote}.
Isto seria mais útil, sendo que estaríamos a ver todo o tráfego gerado por todos os hosts dentro da sala I321.

Ao ler router \textbf{firetux}, não íamos conseguir ver tráfego local.



\section{Conclusão}
\begin{frame}{Conclusão}{}
    \begin{itemize}
        \item Fundamentos teóricos da Tonalidade
        \item Algoritmos de deteção
        \begin{itemize}
            \item Krumhansl-Schmukler
        \end{itemize}
        \item Key Profiles
        \begin{itemize}
            \item Krumhansl
            \item Temperley
            \item etc.
        \end{itemize}
        \item Implementação em Python
        \begin{itemize}
            \item Music21
        \end{itemize}
        \item Análise de precisão
        \begin{itemize}
            \item Melhorias no algoritmo
        \end{itemize}
    \end{itemize}
\end{frame}

%%%%%%%%%%%%%%%%

%put in our emails..
{\aauwavesbg
\begin{frame}[plain,noframenumbering]
  \finalpage{
      {\Large Obrigado e esperamos que tenham gostado!}\\
      \vspace{5mm}
      Diogo Remião - \href{mailto:up201706373@edu.fe.up.pt}{{\tt up201706373@edu.fe.up.pt}}\\
      José Miguel Pinheiro - \href{mailto:up201706172@edu.fe.up.pt}{{\tt up201705172@edu.fe.up.pt}}\\
  }
\end{frame}}
%%%%%%%%%%%%%%%%

\end{document}
