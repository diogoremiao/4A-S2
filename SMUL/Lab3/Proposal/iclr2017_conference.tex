\documentclass{article} % For LaTeX2e
\usepackage{11785_project,times}
\usepackage[hidelinks]{hyperref}
\usepackage{url}


\title{Project Proposal: Key Detection}


\author{
Diogo Remião \\
\texttt{up201706373@edu.fe.up.pt} \\
\AND
Miguel Pinheiro \\
\texttt{up201705172@edu.fe.up.pt} \\
}

\begin{document}


\maketitle

\section{Introduction}
This project consists on the implementation of an algorithm that is able to detect the key in which the song (or excerpt) is being played.
Being this a very common topic in the multimedia research field, there are multiple algorithms for this purpose.

We chose the Krumhansl-Schmuckler Key-Finding algorithm (\citeauthor{krumhansl}).
This algorithm analyses all the notes played and arranges them in a distribution.
Having this information, it then profiles the distribution of the notes and matches it with the key profile with the best similarity.

\section{Software}
To implement the algorithm, we will make use of Python.
The are a number of different implementations in Python making use of different key detection algorithms.

The repository which we will be using is \url{https://github.com/cuthbertLab/music21}

This library has the ability to use specify which algorithm we want to use to detect the key, including the Krumhansl.



\section{Data to be used}
We make use of excerpts of different songs in which the key is constant, in midi extension.
This is an easier data set that enables us to do proof of concept.
Many songs make use of modulation as a composition technique and that would increase the complexity of the project significantly.


\section{Evaluation}
The evaluation of the output of the program can be set in two different parameters (\citeauthor{inproceedings}).

The first is time, in which we evaluate the accuracy of the algorithm based on the length of the sample in seconds.

The second is the accuracy comparing the output considering "understandable" mistakes.
For example, the output might give the minor / major relative of the main key, as well as the dominant or subdominant key.
Comparing the percentage of each category might give good insights of the weaknesses of the algorithm.

\nocite{Gjerdingen}

\bibliography{11785_project}
\bibliographystyle{11785_project}

\end{document}
