\section{Introduction} \label{sec:intro}

Musical key identification is a fundamental part of musical analyses.
When analysing a piece of (tonal) music, the key is a main attribute to be considered \cite{temperley2004cognition}.
It is the basis for further music analyses, such as the mood or emotional connotation it has.

Although an important aspect to consider during the analyses process, musical key identification can only be done by well-trained people,
familiar with the field of composition techniques and analyses.
The common listener does not know (neither cares) in which key the song is being played.
At most, the person might distinguish the key has being minor or major from the mood of the song, although completely abstracted from the theoretical implications it has \cite{dparncutt}.

Nonetheless, like any other field-of-study in STEM, different mathematical algorithms have been created with the objective of automating the process of musical key identification.
We will go through them in the \textbf{Related Work} and \textbf{Method} section.

Finally, in the \textbf{Evaluation} section, we will test the implementation and go through the data we obtained, drawing the necessary conclusions.


