\section{Conclusion} \label{sec:conclusion}

We start by extracting musical information from the MIDI files and compile it into the libraries \textit{music21.stream} class.
After this conversion is done, a distribution profile is created with each pitch in the chromatic scale total usage time.
The Krumhansl algorithm is then applied, where a correlation between previously defined key-profiles for each possible key and the sample are done.
The highest output is then presented as the determined key.

Five key-profiles proposals were tested, with Bellman-Budge key profiles having the best accuracy, with 76.98\% correct answers
The oldest and original key profiles proposed by krumhansl-Schmuckler/Kessler had the worst precision of all, with 68.65\% correct answers.

In terms of neighbouring keys, Temperley-Kostka-Payne had the worst performance, with 24.05\% of the answers not falling in neither the tonic, relative, dominant of parallel keys.
On the other hand, Aarden-Essen key-weightings presented only 16.82\% of answers not falling in neither of the above categories.

In general, the correlation method presented by krumhansl had an average accuracy of 73.466\%.

As far as the comparison with human performance is concerned, with do not have the desired data that to make that analyses.
One study suggested that musically trained people can identify the correct key in 75\% of the cases, tough after listening only to the first measure \cite{cohen1977tonality}.
This experiment was also conducted using samples from \textit{J.S.Bach Well Tempered Clavier}, in which algorithms in this paper had 100\% accuracy but could analyse both the \textit{Prelude \& Fugue} to the full extent.

It is also important to know that in order to compare the types of mistakes and approaches from humans and algorithms, further research would be required.

Concluding, we indicate the following weak points in this project:
\begin{itemize}
    \item Only the Krumhansl correlation algorithm was tested, with other approaches not taken into consideration
    \item The number of samples and its diversity is not enough to do a fair comparison between the different key-weightings. More samples would be required to obtain accuracy levels closer to the actual algorithms performance levels.
    \item The testing was conducted using all the time frame of the pieces. These pieces have different parts that use different keys, so such a simple approach is not fair to evaluate the accuracy of the algorithms. \textit{Rachmaninov's concertos} are a good example of this flaw.
\end{itemize}