\section{Evaluation} \label{sec:evaluation}

\subsection{Data set} \label{sec:data_set}
The first aspect to mention is the data set used.
In order to diversify the test, different sample groups were created and analysed separately.
This is because, if we are to make a fair judgement on the algorithms precision, we need to understand the theoretical implications each music sample has.

The first sample group is \textbf{J.S.Bach's Well Tempered Clavier}.
This is a compilation of 24 pieces written for each of the possible keys.
For this matter, it is an excellent data set as it goes through all possible solutions.

The second sample group in \textbf{F.Chopin's Études}.
Similarly, these 18 pieces have very diversified keys, presenting a good testing opportunity of the algorithm.
However, it is important to mention that Chopin and Bach have very different composition techniques.
This means that both data sets will bring different considerations.

The third sample group is \textbf{S.Rachmaninov's Piano Concertos 2 \& 3}.
Although this data set only has 6 samples (3 for each \textit{concerto}), they are much longer than previous ones.
This means that each sample is much more diverse and has more variations , making it much harder for the algorithm to predict the key.

The forth sample group is \textbf{Contemporary Music}, with 7 pieces including Debussy, Ravel and Satie.
This composers had a different understating of tonality, almost disregarding it.
In other words, given that the models were based on tonal perception, we expect that the algorithm will not do so well in this data set.

The last sample group is \textbf{Pop Music}, with 8 pieces.
This genre differs from the rest of the samples, as it is simpler and repetitive, even amongst themselves.
In fact, the chord progressing is most of the time very simple, consisting on perfect cadences that give away the key instantly.
Therefore, it is expected that the algorithm will do very well it this case.

\subsection{Analyses Considerations} \label{sec:analyses_considerations}
As mentioned before, although we will make use of the \citeauthor{krumhansl2001cognitive} algorithm, different key can be used.
We will test the precision of some of them, which are implemented in library we are using.

The different key profiles we will be testing are the ones mentioned in section \ref{sec:analysis_discrete}.

Not only will we evaluate the precision of the algorithm, i.e., if it predicted the right answer, but also analyse the wrong answers.
As mentioned in \ref{sec:key}, some key profiles are very similar to each other, namely:
\begin{itemize}
    \item Relative
    \item Dominant
    \item Parallel
\end{itemize}

Although still wrong answer, they are understandable mistakes because of the similarity, and analysing the excerpt might gives us some insights on the algorithm's output.

\subsection{Bach Analyses} \label{sec:bach}

\begin{center}
    \begin{tabular}{|c||c c c c c||} 
    \hline
    \% & Krum & Temp & Bell & Aard & Craig \\
    \hline\hline
    Correct & 100 & 100 & 100 & 100 & 100\\ 
    \hline
    Relative & 0 & 0 & 0 & 0 & 0\\
    \hline
    Dominant & 0 & 0 & 0 & 0 & 0\\
    \hline
    Parallel & 0 & 0 & 0 & 0 & 0\\
    \hline
    Other & 0 & 0 & 0 & 0 & 0\\
    \hline
   \end{tabular}
\end{center}

As we can observe from the table above, all the algorithms have 100\% accuracy. This is more-less to be expected.
The composition style in this pieces is very uniform. 
Perfect cadences are executed multiple times during the piece.
Perfect cadences correspond to the chord progression I - IV - V -I, which have all the major pitches in that key.
This way, because this pitches have a much higher weight in the analyses it will be easier to identify the key.

Another aspect to take consideration is modulations. Modulations consists in changing the main key of a section of the piece.
Bach often makes use of a modulation technique called \textit{Circle of fifths}.
This technique enables to modulation to many keys seamlessly to closely related keys.
These keys share many notes, and therefore do not impact much the prediction unless they are used very often.

Preludes should also be easier to analyse than Fugues, and in this case they were analysed together.

\subsection{Chopin Analyses} \label{sec:chopin}

\begin{center}
    \begin{tabular}{|c||c c c c c||} 
    \hline
    \% & Krum & Temp & Bell & Aard & Craig \\
    \hline\hline
    Correct & 94.44 & 94.44 & 94.44 & 94.44 & 100\\ 
    \hline
    Relative & 0 & 5.56 & 5.56 & 0 & 0\\
    \hline
    Dominant & 5.56 & 0 & 0 & 0 & 0\\
    \hline
    Parallel & 0 & 0 & 0 & 0 & 0\\
    \hline
    Other & 0 & 0 & 0 & 5.56 & 0\\
    \hline
   \end{tabular}
\end{center}

Although from a different age than Bach (Baroque), Chopin (Romantic) composed based on tonal rules and very fluently melodies and chords.
These meant that modulating were once again often used with closely related keys, like Dominant, sub-dominant and relative keys.
In these pieces in particular, modulating sections were long.

This explains while we see some predictions falling in other categories (the "other" prediction in "Aarden" was in fact to the sub-dominant).
It also needs to be taken into account that \textit{Études} are composed to train technique, often being simpler in terms of melody and harmonization.
These factors of course make it easier for the algorithm to work.


\subsection{Rachmaninov Analyses} \label{sec:rach}

\begin{center}
    \begin{tabular}{|c||c c c c c||} 
    \hline
    \% & Krum & Temp & Bell & Aard & Craig \\
    \hline\hline
    Correct & 16.67 & 33.33 & 33.33 & 33.33 & 33.33\\ 
    \hline
    Relative & 0 & 0 & 0 & 0 & 0\\
    \hline
    Dominant & 0 & 0 & 0 & 16.67 & 0\\
    \hline
    Parallel & 16.67 & 0 & 0 & 0 & 0\\
    \hline
    Other & 66.67 & 66.67 & 66.67 & 50 & 66.67\\
    \hline
   \end{tabular}
\end{center}

Rachmaninov piano concertos are the climax of the romantic period.
A period defined by emotions and a megalomaniacal Russian culture result in very intense and diverse music,
with these piano concertos being arguably the best representation of that.
Being such long pieces, we can identify different parts using different keys, techniques and tempos.
We can define these pieces as literally "All over the place" in terms on music composition.

Therefore, it is understandable the difficulty the algorithms have in correctly predicting the answer.
This category was in fact selected precisely to prove the major flaw in Krumhansl approach: no segmentation of the pieces.
When you have a long piece that modulates into different keys for long periods of time, it is impossible to predict the main key using this algorithm.
In fact it does not even make any sense to analyse these pieces in such a generic way, as their complexity calls for a more local approach.


\subsection{Contemporary Analyses} \label{sec:contemp}

\begin{center}
    \begin{tabular}{|c||c c c c c||} 
    \hline
    \% & Krum & Temp & Bell & Aard & Craig \\
    \hline\hline
    Correct & 57.14 & 57.14 & 57.14 & 42.86 & 57.14\\ 
    \hline
    Relative & 0 & 0 & 0 & 14.29 & 0\\
    \hline
    Dominant & 14.29 & 14.29 & 14.29 & 14.29 & 14.29\\
    \hline
    Parallel & 0 & 0 & 0 & 0 & 0\\
    \hline
    Other & 28.57 & 28.57 & 28.57 & 14.29 & 28.57\\
    \hline
   \end{tabular}
\end{center}

Contemporary music stands out for not following formal tonal music rules.
Although the pieces chosen are still from the early part of the movement, we can already identify, for example, that modulations do not make use of any technique in particular.
The composer bases its composition in the sonic effect rather than the formal modulation techniques.
For example, cadences are rarely perfect. In fact, tonal chords progressions are very rare.

As we are applying a tonal algorithm to "border-tonal" music, we can accept the 50\% accuracy obtained as a good performance from the algorithms. 
Especially considering that most mistakes were understandable, predicting the dominant of relative keys which, as explained in section \ref{sec:key}, have very similar profiles.
\subsection{Pop Analyses} \label{sec:pop}

\begin{center}
    \begin{tabular}{|c||c c c c c||} 
    \hline
    \% & Krum & Temp & Bell & Aard & Craig \\
    \hline\hline
    Correct & 75 & 75 & 100 & 100 & 87.5\\ 
    \hline
    Relative & 0 & 0 & 0 & 0 & 0\\
    \hline
    Dominant & 25 & 0 & 0 & 0 & 0\\
    \hline
    Parallel & 0 & 0 & 0 & 0 & 0\\
    \hline
    Other & 0 & 25 & 0 & 0 & 12.5\\
    \hline
   \end{tabular}
\end{center}

Pop music, as mentioned before, is very simple and makes use of repetitive chord progressions.
Modulations are rarely made, and when they are, it is using closely related keys.
Because these keys all share most pitches in common, some mistakes are understandable.
However some algorithms got 100\% accuracy, proving that in fact, this is the easiest genre to analyse.
Bach might seem to be better, but Bach's sample list was bigger than Pop's.
Therefore a direct comparison is not possible.
As argument can be made given the fact that analysing by hand a piece of Bach might take a day, while Pop music takes 1 minute.



\subsection{Profile Comparison} \label{sec:total_analyses}

\begin{center}
    \begin{tabular}{|c||c c c c c||} 
    \hline
    \% & Krum & Temp & Bell & Aard & Craig \\
    \hline\hline
    Correct & 68.65 & 71.98 & 76.98 & 74.13 & 75.59\\ 
    \hline
    Relative & 0 & 1.11 & 1.11 & 2.86 & 0\\
    \hline
    Dominant & 8.96 & 2.86 & 2.86 & 6.19 & 2.86\\
    \hline
    Parallel & 3.33 & 0 & 0 & 0 & 0\\
    \hline
    Other & 19.06 & 24.05 & 19.05 & 16.82 & 21.55\\
    \hline
   \end{tabular}
\end{center}

Comparing the average performance of the different algorithms, the original algorithm, Krumhansl, has the worst performance,
while Bellman's key profiles have the best results by a significant margin.

The original key profiles were based completely in a subjective analyses done with volunteers based on "sounding nice".
Although this might work most of the time, it only goes so far, as complex musical properties come into play when the excerpts are more complex.
It is expected that, with more technology and more research, newer models were suggested after studying the original model and identifying its flaws.

Another aspect to take into consideration is the type of weights given to each key.
In Krumhansl, the tonic has a weight of 6.35, with the next pitch (not part of the scale of that key) having a weight of 2.23.
In Aarden, the tonic weights 17.7661, with next pitch weighting 0.145624.

We can see that in Aarden / Bellman profiles, much bigger importance is given to the notes that are part of the scale of that pitch, in contrast with Krumhansl / Temperley.
This reduces the error caused by ornamentations of passing notes.
Simple (Craig) profiles give 0 weight to pitches that are not part of the scale and 1 to the ones that are, with the tonic weighting 2.

Given that the sample list is not that extended and/or diverse, the algorithms can be said to be performing at more less the same level,
with the revisions slightly outperforming in some more complex categories.