\section{Related Work} \label{sec:related}

Like mentioned before, different algorithms were developed with the aim of, via a mathematical model, detect the musical key of a piece of music.
However, the first aspect to take into considering is the type of analyses we are doing.

One would be to extract directly from an audio sample.
This is however not the most common method has it involves creating chromagram to extract which notes are being played in each instant \cite{pauws2004musical}.
This is a more complex method and key detection from musical audio solutions are still limited.
Furthermore, in the evaluation section, the precision of the extraction of pitches would also have to be considered, adding another complexity layer to the problem.

We therefore opted for extracting the data from MIDI files, where we make use of symbolic data, and therefore obtain ground-true information related to the piece of music \cite{Raffel2016ExtractingGI}.
This will further considered in the \textbf{Implementation} section.

Having extracted the data, the next step is to apply and algorithm that outputs a prediction.
The algorithm we will be making use of is the \textbf{Maximum Key-profile Correlation} (MKC) \cite{krumhansl2001cognitive} based on key profiles. This algorithm will be further analysed in the \textbf{Method} section.

A number of studies have made use of this algorithm, or created variations of it, to compute the musical key.
\cite{Temperley2004Musicp} proposed revised versions of the key profiles initially presented by Krumhansl \cite{krumhansl2001cognitive} in order to improve the accuracy of the algorithm.
\cite{harjagraph} identifies a possible inaccuracy of the algorithm related to the window size choice, offering as a solution applying filters to smooth out local oscillations and impulses that might affect the accuracy of the algorithm.

Some studies have also used this algorithm for other purposes, like in \cite{takeuchi1994maximum} where MKC is used to measure the tonality, colloquially know as "sounding nice", of a piece of music.
\cite{lee2006automatic} made use of MKC for chord recognition. This is also based on profiles, with one for each chord just like we would have one for each key.

There are alternatives to the Krumhansl-Schmukler algorithm, like in \cite{chew2000towards},
where the author makes use of a geometrical representation called the Spiral Array, 
in which pitches are represented by points on a spiral and determines key with a Center of Effect Generator method.