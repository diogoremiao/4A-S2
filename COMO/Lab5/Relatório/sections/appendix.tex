\chapter{Apendice} \label{int}
\section{Script para o 1º e 3º estudo} \label{ex1}
\begin{lstlisting}[language=Python, label=py_script, basicstyle=\fontsize{10.5}{12.5}\selectfont]

	#include "ns3/core-module.h"
	#include "ns3/network-module.h"
	#include "ns3/mobility-module.h"
	#include "ns3/lte-module.h"
	#include "ns3/config-store.h"
	#include <ns3/buildings-helper.h>
	using namespace ns3;
	
	int main (int argc, char *argv[])
	{
	  Time simTime = MilliSeconds (1050);
	  bool useCa = false;
	  int CaNumber=2;  
	  int distance=1000;  
	  CommandLine cmd (__FILE__);
	  cmd.AddValue ("simTime", "Total duration of the simulation", simTime);
	  cmd.AddValue ("useCa", "Whether to use carrier aggregation.", useCa);
	  cmd.AddValue ("CaNumber", "Whether to use carrier aggregation.", CaNumber);
	  cmd.AddValue ("distance", "Total distance of the simulation", distance);
	  cmd.Parse (argc, argv);

	  ConfigStore inputConfig;
	  inputConfig.ConfigureDefaults ();
	  // Parse again so you can override default values from the command line
	  cmd.Parse (argc, argv);
	
	  if (useCa)
	   {
		 Config::SetDefault ("ns3::LteHelper::UseCa", BooleanValue (useCa));
		 Config::SetDefault ("ns3::LteHelper::NumberOfComponentCarriers", UintegerValue (CaNumber));
		 Config::SetDefault ("ns3::LteHelper::EnbComponentCarrierManager", StringValue ("ns3::RrComponentCarrierManager"));
	   }
	
	  Ptr<LteHelper> lteHelper = CreateObject<LteHelper> ();
	  // Create Nodes: eNodeB and UE
	  NodeContainer enbNodes;
	  NodeContainer ueNodes;
	  enbNodes.Create (1);
	  ueNodes.Create (1);
	  // Install Mobility Model
	  MobilityHelper mobility;
	  Ptr<ListPositionAllocator> positionAlloc = CreateObject<ListPositionAllocator> ();
	  positionAlloc->Add (Vector (0.0, 0.0, 0.0)); 
	  mobility.SetPositionAllocator (positionAlloc);
	  
	  mobility.SetMobilityModel ("ns3::ConstantPositionMobilityModel");
	  mobility.Install (enbNodes);
	  //BuildingsHelper::Install (enbNodes);
	  
	  Ptr<ListPositionAllocator> positionAlloc2 = CreateObject<ListPositionAllocator> ();
	  positionAlloc2->Add (Vector (0.0, distance , 0.0)); 
	  mobility.SetPositionAllocator (positionAlloc2);
	  mobility.SetMobilityModel ("ns3::ConstantPositionMobilityModel");
	  mobility.Install (ueNodes);
	  //BuildingsHelper::Install (ueNodes);
		
	  // Create Devices and install them in the Nodes (eNB and UE)
	  NetDeviceContainer enbDevs;
	  NetDeviceContainer ueDevs;
	  // Default scheduler is PF, uncomment to use RR
	  //lteHelper->SetSchedulerType ("ns3::RrFfMacScheduler");
	
	  enbDevs = lteHelper->InstallEnbDevice (enbNodes);
	  ueDevs = lteHelper->InstallUeDevice (ueNodes);
	
	  // Attach a UE to a eNB
	  lteHelper->Attach (ueDevs, enbDevs.Get (0));
	
	  // Activate a data radio bearer
	  enum EpsBearer::Qci q = EpsBearer::GBR_CONV_VOICE;
	  EpsBearer bearer (q);
	  lteHelper->ActivateDataRadioBearer (ueDevs, bearer);
	  lteHelper->EnableTraces ();
	  Simulator::Stop (simTime);
	  Simulator::Run ();
	  Simulator::Destroy ();
	  return 0;
	}

\end{lstlisting}

\section{Script para o 2º estudo} \label{ex2}
\begin{lstlisting}[language=Python, label=py_script, basicstyle=\fontsize{10.5}{12.5}\selectfont]
	#include "ns3/core-module.h"
	#include "ns3/network-module.h"
	#include "ns3/mobility-module.h"
	#include "ns3/lte-module.h"
	#include "ns3/config-store.h"
	#include <ns3/buildings-helper.h>
	#include <cstring>
	#include <string>
	#include <math.h>
	#define PI 3.14159265
	using namespace ns3;
	int main (int argc, char *argv[])
	{
	  Time simTime = MilliSeconds (5000);
	  bool useCa = false;	
	  int distance=100;
	  int nr_of_ue = 2;

	  CommandLine cmd (__FILE__);
	  cmd.AddValue ("simTime", "Total duration of the simulation", simTime);
	  cmd.AddValue ("useCa", "Whether to use carrier aggregation.", useCa);
	  cmd.AddValue ("distance", "Total distance of the simulation", distance);
	  cmd.AddValue ("nr_of_ue", "Total distance of the simulation", nr_of_ue);
	  cmd.Parse (argc, argv);
	  ConfigStore inputConfig;
	  inputConfig.ConfigureDefaults ();
	  // Parse again so you can override default values from the command line
	  cmd.Parse (argc, argv);
	
	  if (useCa)
	   {
		 Config::SetDefault ("ns3::LteHelper::UseCa", BooleanValue (useCa));
		 Config::SetDefault ("ns3::LteHelper::NumberOfComponentCarriers", UintegerValue (2));
		 Config::SetDefault ("ns3::LteHelper::EnbComponentCarrierManager", StringValue ("ns3::RrComponentCarrierManager"));
	   }
	
	  Ptr<LteHelper> lteHelper = CreateObject<LteHelper> ();
	  // Create Nodes: eNodeB and UE
	  NodeContainer enbNodes;
	  NodeContainer ueNodes;
	  enbNodes.Create (1);
	  ueNodes.Create (nr_of_ue);
	  // Install Mobility Model
	  MobilityHelper mobility;
	 Ptr<ListPositionAllocator> positionAlloc = CreateObject<ListPositionAllocator> ();
	  positionAlloc->Add (Vector (0.0, 0.0, 0.0)); //node 0 Sink node
	  mobility.SetPositionAllocator (positionAlloc);
	  mobility.SetMobilityModel ("ns3::ConstantPositionMobilityModel");
	  mobility.Install (enbNodes);
	  //BuildingsHelper::Install (enbNodes); 
	  double current_angle = 0.0;
	  double angle_step = 360.0 / nr_of_ue;
	  Ptr<ListPositionAllocator> positionAlloc2 = CreateObject<ListPositionAllocator> ();
	  while (current_angle < 360.0 )
		{
			//std::cout << "Current angle: " << current_angle << "\n";
			positionAlloc2->Add (Vector (distance*cos(current_angle * PI / 180.0), distance*sin(current_angle * PI / 180.0), 0.0)); //Sender node x
			current_angle += angle_step;
		}
	  mobility.SetPositionAllocator (positionAlloc2);
	  mobility.SetMobilityModel ("ns3::ConstantPositionMobilityModel");
	  mobility.Install (ueNodes);
	  //BuildingsHelper::Install (ueNodes);
	  // Create Devices and install them in the Nodes (eNB and UE)
	  NetDeviceContainer enbDevs;
	  NetDeviceContainer ueDevs;
	  // Default scheduler is PF, uncomment to use RR
	  //lteHelper->SetSchedulerType ("ns3::RrFfMacScheduler");	
	  enbDevs = lteHelper->InstallEnbDevice (enbNodes);
	  ueDevs = lteHelper->InstallUeDevice (ueNodes);
	  // Attach a UE to a eNB
	  lteHelper->Attach (ueDevs, enbDevs.Get (0));
	 // Activate a data radio bearer
	  enum EpsBearer::Qci q = EpsBearer::GBR_CONV_VOICE;
	  EpsBearer bearer (q);
	  lteHelper->ActivateDataRadioBearer (ueDevs, bearer);
	  lteHelper->EnableTraces ();
	  Simulator::Stop (simTime);
	  Simulator::Run ();	
	  Simulator::Destroy ();
	  return 0;
	}

\end{lstlisting}
