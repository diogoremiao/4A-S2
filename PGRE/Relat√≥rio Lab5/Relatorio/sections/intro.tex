\chapter{Introdução} \label{sec:intro}

O objetivo de trabalho prende-se com a análise de \textbf{ferramentas de gestão e monitorização} de serviços de uma rede.
Ao contrário de ferramentas como MRTG e NTOP que monitorizam o tráfego, neste trabalho serão abordadas ferramentas que monitorizam diretamente os diferentes sistemas e os serviços neles alojados.

As ferramentas utilizadas são o \textbf{Nagios Core} e \textbf{Zabbix}, ambas grátis e open-source.
Será igualmente realizada uma análise às diferentes funcionalidades e capacidade de personalização de ambas as ferramentas num ambiente de teste criada no nossa bancada, onde serão alojados vários serviços nos diferentes computadores.
Testes de falha de sistemas e serviços serão efetuados de modo a analisar o funcionamento das ferramentas na deteção de falhas.

Por fim, será feita uma análise comparativa entre estas ferramentas com duas outras alternativas no mercado.

