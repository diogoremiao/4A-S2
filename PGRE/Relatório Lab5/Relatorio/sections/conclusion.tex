\chapter{Conclusão}

Neste trabalho, começamos por configurar diferentes serviços na bancada, nomeadamente servidores DNS, NTP, FTP, HTTP e Email.
Nos dispositivos de rede, o switch e router, foi ativado o protocolo SNMP.

Procedeu-se à configuração quer do \textbf{Nagios}, quer do \textbf{Zabbix}.
Estas ferramentas foram configuradas com o intuito de testar a disponibilidade quer dos próprios sistemas na rede,
quer dos serviços neles alojados.

Concluíu-se que o processo de configuração no Zabbix é mais simples e intuitivo do que no Nagios.
Isto deve-se ao facto da configuração no Zabbix ser feita na sua interface Web, ao ponto que no Nagios é feito diretamente editando ficheiros de configuração no sistema.
No entanto, os \textbf{plugins} do Nagios provaram ser mais versáteis e diversificados do que os \textbf{items} do Zabbix, aumentando a especificidade dos testes executados na rede.

Após o teste de bom funcionamento dos serviços, foram provocadas falhas nos serviços e sistemas,
e analisado os comportamentos das ferramentas. Ambas as ferramentas apresentaram a informação de forma clara.

Todavia, a interface do Zabbix é visualmente mais apelativa, devido à possibilidade de criar \textbf{dashboards} altamente customizáveis,
onde o utilizador pode criar mapas interativos que esquematicamente representam a rede e o status dos seus sistemas e serviços.

Finalmente, fez-se uma pequena análise de outras duas ferramentas de monitorização.

O \textbf{Grafana} provou seu uma ferramenta de monitorização poderosa devido à sua UI baseada em gráficos temporais e \textit{gauges}.
Mostra assim ser uma boa ferramenta complementar para apresentar informação high-level ao utilizador.

O \textbf{openDCIM} é uma ferramenta vocacionada para monitorização de \textit{data centers}, com algumas funcionalidades low-level não oferecidas por outras ferramentas.
No entanto, fica atrás a nível de UI, assim como uso para generalizado para testes de serviços devido à sua fraca oferta de plugins.
