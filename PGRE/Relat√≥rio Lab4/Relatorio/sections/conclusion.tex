\chapter{Conclusão} \label{sec:conclusion}

Neste trabalho criamos uma tipologia de rede onde nos foi possível implementar o protocolo de \textit{routing} interno \textbf{OSPF}.
Avaliamos as diferentes rotas e os pesos atribuídos a cada uma.
Testamos também um cenário de falha de uma ligação, onde o protocolo foi capaz de se adaptar e criar uma nova rota  por outro caminho.

Numa segunda fase, colaboramos com os restantes colegas para implementar o protocolo de \textit{routing} externo \textbf{BGP}.
Com implementação do OSPF na nossa bancada criamos efetivamente um \textbf{Autonomous System}. As restantes bancadas também tinham a sua AS.

Foram criados 6 AS dentro da sala, onde em cada bancada foi configurado um ABR a correr o protocolo BGP.
Foi possível observar os diferentes caminhos para os redes dentro de cada AS.
Consequentemente, configuramos os routers de modo a modificar as rotas escolhidas pelo BGP para o tráfego externo.

Em suma, adquirimos conhecimentos a nível do funcionamento de protocolos de \textit{routing}, assim como na sua implementação numa configuração de rede laboratorial.
Também foi possível configurar protocolos de \textit{routing} numa SDN, transformando os computadores em routers.
