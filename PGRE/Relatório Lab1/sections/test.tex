\chapter{Test Environment} \label{test}
Nesta secção vamos explicar como se procedeu para obter os resultados neste trabalho.

\section{Setup}
Tal como indicado previamente, os dois servidores web estão alojados no tux 172.16.1.13, nas portas 80 e 81.

O proxy Squid tem por predefinição a porta 80 aberta mas a porta 81 teve que ser manualmente aberta no ficheiro de configuração do Squid,
de forma a permitir o acesso ao segundo site.

Os servidores web foram acedidos periodicamente por um outro tux dentro da mesma rede, o 172.16.1.14.

\section{Logging e Cronjobs}
Os logs de acesso do Apache irão mostrar diferentes tipos de acessos aos dois websites.
Em primeiro lugar, foram feitos acessos de um computador externo ligado pela rede VPN. Estes acessos foram feitos com o intuito
de testar o funcionamento das diferentes ferramentas.

Para criar acessos periódicos que criem logs com mais informação útil, foram configurados acessos periódicos do tux 172.16.1.14.
Este acessos ocorriam inicialmente uma vez por hora durante um dia e 4 vezes por hora no segundo dia.
Os acessos eram feitos quer diretamente, quer através do proxy, de forma a gerar mais informação.
Para este efeito forma adicionados Cronjobs à crontab do tux, que são nada mais nada menos que comandos executados pelo computador
de forma autónoma nos \textit{timestamps} desejados

A razão pela qual não foram gerados mais logs foi o facto de no inicio da semana estes Cronjobs terem sido configurados para correr, 
e devido a problemas técnicos, houve um wipeout de vários serviços, nomeadamente do proxy. Tivemos então que recomeçar o teste mas com
menos tempo para gerar os logs devido à data limite de entrega do trabalho. De qualquer modo, a informação gerada foi suficiente
para obter resultados úteis e proceder à sua análise.