\chapter{Conclusão}
Em suma, ambas as ferramentas são muito poderosas e apresentação uma boa análise estatística dos logs gerados. 
Pelos motivos acima referenciados, o AWStats aparenta ser a melhor solução, principalmente pela capacidade de servir vários websites ao
mesmo tempo e de puder dar \textit{reload} à informação diretamente da página.
O AWStats também apresenta mais informação que Webalizer que o torna uma ferramenta mais poderosa.

O Webalizer por outro lado apresenta uma setup e uma interface mais simples, que o torna uma boa ferramenta para testes menos intensivos.

Existem outras soluções para análise de logs como é o caso de W3Perl. Uma desvantagem do W3Perl é o facto de ter deixado de receber suporte em 2015, o que torna a ferramenta obsoleta.

Um aspeto que gostávamos de ter melhorado neste trabalho é a geração de logs.
Não fomos muito criativos na geração destes logs e eles não tem uma extensão temporal em que os gráficos apresentação sejam mais interessantes.
Podíamos ter sido criados gráficos a simular uma utilização mais realista com utilização mais intensiva a meio da manhã por parte de um host e mais intensiva de tarde por parte de outro.
Conseguimos no entanto aprender a utilizar estas ferramentas para analisar o impacto nos recursos de infraestrutura que alojar um site apresenta.
Adquirimos também conhecimento sobre o funcionamento de um proxy e qual a sua utilidade em contexto real, especialmente neste trabalho onde a sua utilização foi especialmente necessária para podermos observar os resultados obtidos.
