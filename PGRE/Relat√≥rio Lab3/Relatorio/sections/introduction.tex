\chapter{Introdução}

Este trabalho tem como objetivos a compreensão de requisitos de endereçamento IP numa rede empresarial e configuração de servidores DNS locais.
Para tal, foi desenvolvido um projeto de endereçamento tendo em conta os diferentes sub-sistemas da rede empresarial,
nomeadamente a criação de uma intranet local, sem acesso à internet, e uma DMZ, que funciona como a fronteira para o exterior da empresa, com acesso à internet.
Todas estes sub-sistemas terão a sua própria rede local, pelo que será necessário configurar diferentes Vlans para este efeito.

Depois do plano endereçamento, irá ser abordada a configuração de servidores DNS dentro da empresa, com o objetivo de resolver domínios locais.
Estes endereços apenas poderão ser visíveis nas redes locais permitidas, pelo que será utilizada uma bordagem de \textit{Split DNS} para filtrar os pedidos.

Por fim, iremos configurar a rede de forma a que apenas a DMZ tenha acesso à internet, através do protocolo NAT.

%No final de trabalho, esperamos ter mais conhecimento na estruturação de um rede empresarial nomeadamente a nível de endereçamento.
%Também esperamos melhorar o nosso domínio no DNS, quer a nível de compreensão do seu funcionamento, quer a nível da criação de servidores para redes locais.
%Por fim, ainda que apenas uma breve introdução, também esperamos compreender o funcionamento geral do protocolo NAT, responsável pela ligação da rede à internet.